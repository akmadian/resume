%%%%%%%%%%%%%%%%%%%%%%%%%%%%%%%%%%%%%%%%%
% Medium Length Professional CV - RESUME CLASS FILE
%
% This template has been downloaded from:
% http://www.LaTeXTemplates.com
%
% This class file defines the structure and design of the template. 
%
% Original header:
% Copyright (C) 2010 by Trey Hunner
%
% Copying and distribution of this file, with or without modification,
% are permitted in any medium without royalty provided the copyright
% notice and this notice are preserved. This file is offered as-is,
% without any warranty.
%
% Created by Trey Hunner and modified by www.LaTeXTemplates.com
%
%%%%%%%%%%%%%%%%%%%%%%%%%%%%%%%%%%%%%%%%%

\documentclass{resume} % Use the custom resume.cls style

\usepackage[left=0.75in,top=0.6in,right=0.75in,bottom=0.6in]{geometry} % Document margins
\usepackage[colorlinks = true,
            linkcolor = blue,
            urlcolor = blue,
            citecolor = blue,
            anchorcolor = blue]{hyperref}
\newcommand{\changeurlcolor}[1]{\hypersetup{urlcolor=#1}}      
\newcommand{\tab}[1]{\hspace{.2667\textwidth}\rlap{#1}}
\newcommand{\itab}[1]{\hspace{0em}\rlap{#1}}
\name{Ari Madian}
\address{{akmadian@gmail.com - (253)249-6636}}
\address{
    \href{https://www.linkedin.com/in/arimadian/}{linkedin.com/in/arimadian} -
    \href{https://github.com/akmadian}{github.com/akmadian}
}

\begin{document}
\begin{rSection}{Education}

{\bf University of Washington} \hfill { Est. Graduation Jun 2022 } 
\\ B.S. Informatics\hfill { } %GPA

{\bf South Seattle College} \hfill { Graduated Jun 2020 } 
\\ A.S. Computer Science \hfill {GPA: 3.6}
\\ \small Activities: Founded Hour of Code Club, Phi Theta Kappa Member


\end{rSection}
\begin{rSection}{Skills}
    \begin{tabular}{ @{} >{\bfseries}l @{\hspace{6ex}} l }
        Programming Languages &  \small Python, C\#, Javascript, TypeScript, Java, C++, R, MATLAB\\
        Software \& Tools & \small AWS, Git, Node.js, Vue.js, React.js, CSS, HTML, SQL, Docker\\
        & Excel, Tableau, Jupyter, .NET, Linux and Windows
    \end{tabular}
\end{rSection}
\begin{rSection}{Experience}
    \begin{rSubsection}
        {Open Source Community Coordinator (Contract)}
        {Mar 2020 - Jun 2020}
        {\em Creative Commons}
        {}
        \begin{itemize}
            \setlength\itemsep{-0.2em}
            \item\small Managed Creative Commons' open source presence and a 600+ person open source community spanning six continents and 20 time zones. Oversaw a 30\% growth in community size during my time.
            \item Drafted and launched a unique community engagement initiative to reward and retain high quality open source contributors that saw 30 actively engaged participants within two months of launch.
            \item Reviewed hundreds of PRs and issues; maintained social media accounts, the CC open source site and public wiki; significantly reduced repository maintenance overhead by automating workflows with GitHub actions.
            \item Mentored a diverse set of 125 interns and internship applicants through Google Summer of Code and Outreachy.
        \end{itemize}
    \end{rSubsection}
    \begin{rSubsection}
        {Google Summer of Code Student (Intern)}
        {May 2019 - Aug 2019}
        {\em Creative Commons}
        {[\href{https://github.com/creativecommons/chooser}{GitHub Repo}][\href{https://chooser-beta.creativecommons.org/}{Live Site}]}
        \begin{itemize}
            \setlength\itemsep{-0.25em}
            \item \small Made free and simple copy(right/left) licensing more accessible to millions by revamping an outdated website with a focus on UX, accessibility, and an educational experience of Creative Commons licenses.
            \item Created wireframes and UI/UX prototypes, conducted usability testing and accessibility audits.
            %\item Continuing work on the project to this day via personal contributions and mentoring with Outreachy.
            \item Site built with Vue/ VueX, Jest and Nightwatch for testing, along with i18n for internationalization.
        \end{itemize}
    \end{rSubsection}
\end{rSection}
\begin{rSection}{Projects}
    \begin{rSubsection}
        {NZXTSharp}
        {Dec 2018 - Oct 2019}
        { \normalsize An open source .NET SDK for NZXT Devices }
        {[\href{https://github.com/akmadian/NZXTSharp}{GitHub Repo}]}
        \begin{itemize}
            \setlength\itemsep{-0.25em}
            \item\small Designed and built an efficient and scalable SDK that controls NZXT PC components with C\# by teaching myself about, reverse engineering, and emulating serial and HID communication protocols.
            \item \textasciitilde 3.5k downloads on \href{https://www.nuget.org/packages/NZXTSharp}{nuget.org}; 39 Stars, 13 Forks on GitHub
        \end{itemize}
    \end{rSubsection}
    \begin{rSubsection}
        {OpenFinance}
        {Aug 2020 - Present}
        { \normalsize Open source personal finance software for my own use. }
        {[\href{https://github.com/akmadian/openfinance}{GitHub Repo}]}
        \begin{itemize}
            \setlength\itemsep{-0.2em}
            \item \small Developed an alternative to commercial finance software to manage and visualize budgets and spending.
            \item Can import bank statements, pull account positions from Coinbase and TD Ameritrade using their APIs, amd aggregate stock news, sentiment, and charts with TradingView.
            \item Built with Flask and SQLite backend, with a React/ Redux frontend.
        \end{itemize}
    \end{rSubsection}
\end{rSection}
\end{document}
